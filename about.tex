
project/
│
├── index.html
├── room.html
├── styles.css
└── scripts.js


1. ملف HTML الرئيسي: index.html
الوصف: الصفحة الرئيسية للموقع التي تعرض خيار الدردشة العامة وإنشاء غرفة دردشة خاصة.

3. ملف CSS لتنسيق الموقع: styles.css
الوصف: ملف CSS لتحسين مظهر الموقع وجعله أكثر جاذبية.

4. ملف JavaScript: scripts.js
الوصف: ملف JavaScript يحتوي على المنطق اللازم لإدارة الدردشة العامة وإنشاء الغرف الخاصة.


_________________________________________________________________________________
كيف يعمل المشروع:
الدردشة العامة:

عند الضغط على زر "الدردشة العامة"، يتم الانتقال إلى صفحة الدردشة 
(room.html) مع التحميل الافتراضي للرسائل المخزنة في التخزين المحلي.
إنشاء غرفة خاصة:

عند الضغط على زر "إنشاء غرفة خاصة"، يتم توليد معرف 
غرفة فريد والانتقال إلى صفحة الدردشة مع تضمين معرف الغرفة في الرابط.
 يمكن للمستخدم نسخ الرابط ومشاركته مع أصدقائه للانضمام إلى نفس الغرفة.
التخزين المحلي:

يتم تخزين الرسائل الخاصة بكل غرفة في التخزين المحلي للمتصفح باستخدام معرف الغرفة
 كمفتاح. عند الدخول إلى نفس الغرفة مرة أخرى، يتم تحميل الرسائل المخزنة.
ملاحظات:
حذف الرسائل: جميع الرسائل مؤقتة وتُحذف تلقائيًا عند إغلاق أو تحديث المتصفح.
قابلية التطوير: يمكن تطوير الموقع لاحقًا ليتضمن ميزات مثل حذف الرسائل،

تحديد المستخدمين، أو حتى تسجيل الدردشة في قاعدة بيانات حقيقية إذا لزم الأمر.
بهذه الطريقة
، يصبح الموقع منصة للدردشة المؤقتة مع إمكانية إنشاء غرف دردشة خاصة بسهولة.